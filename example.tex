\def\OPTIONConf{0}%
\def\OPTIONArxiv{0}%
%
% \documentclass{article}
\documentclass{purple}
% `purple' is an experimental class with unusual formatting

\usepackage{mathpartir}
\usepackage{amsmath}
\usepackage{marvosym}  % for \Pointinghand / \Hand in llproof

\usepackage{srcltx}
\usepackage{goodcharter}
\usepackage{euler}

\usepackage{jdunfield}
\usepackage{llproof}
\usepackage{rulelinks}



\newcommand{\macroname}[1]{\texttt{\Backslash}\textvtt{#1}}

\newcommand{\unitty}{\tyname{1}}
\newcommand{\rulename}[1]{\textsf{#1}}
\newcommand{\elabtyperulename}[1]{\rulename{\textcolor{dDigPurple}{E#1}}}

\newrulecommand{EVar}{\elabtyperulename{var}}



\title{Example}

\author{Jana Dunfield}


\begin{document}
\maketitle

Currently, documentation is only in TeX comments inside the various \textvtt{.sty} and \textvtt{.cls} files.

\section{rulelinks.sty}

\begin{figure}[htbp]
  \centering

  \judgbox{\Gamma \entails e : A}{Under typing assumptions $\Gamma$, expression $e$ has type $A$}
  \begin{mathpar}
      \centerruleplaceholder{
           \Infer{\EVar}
                 {}
                 {\Gamma, x : A, \Gamma' \entails x : A}
      }{
          No subsumption rule,
          \\
          because I hate freedom
     }
  \end{mathpar}
  
  \caption{Typing}
  \label{fig:typing}
\end{figure}

Using the macro \macroname{EVar} for the first time anchors a hyperlink pointed to
by subsequent uses: \EVar.



\clearpage

\section{llproofs.sty}

\newcommand{\ctxoutsym}{{\dashv}}
\newcommand{\ctxout}[1]{\mathrel{\ctxoutsym}{#1}}

\newcommand{\judgetp}[2]{{#1} \entails {#2}}
\newcommand{\judgetpPf}[3]{\Pf{#1}{\entails}{#2}{#3}}

\newcommand{\declsynjudg}[3]{{#1} \entails {#2} \Rightarrow {#3}}
\newcommand{\declsynjudgPf}[4]{\ePf{#1}{{#2} \Rightarrow {#3}}{#4}}
\newcommand{\synjudg}[4]{{#1} \entails {#2} \Rightarrow {#3} \ctxout{#4}}
\newcommand{\synjudgPf}[5]{\ePf{#1}{{#2} \Rightarrow {#3} \ctxout{#4}}{#5}}

\newcommand{\substextendsym}{\longrightarrow}
\newcommand{\substextend}[2]{{#1} \substextendsym {#2}}
\newcommand{\substextendPf}[3]{\Pf{#1}{\substextendsym}{#2}{#3}}

\newrulecommand{DeclVar}{\rulename{DeclVar}}
\newrulecommand{Var}{\rulename{Var}}

\newcommand{\FEV}[1]{\mathsf{FEV}(#1)}  % free existential variables

This example of the \textvtt{llproof} environment is adapted from
Dunfield and Krishnaswami, ICFP 2013.  The source for this file
(example.tex) includes definitions of per-judgment macros, as well
as the \macroname{DerivationProofCase} macro.

The relevant part of the statement being proved is:

\begin{quote}
   Given $\substextend{\Gamma}{\Omega}$ and $\judgetp{\Gamma}{A}$:

     If $\declsynjudg{[\Omega]\Gamma}{e}{A}$
     \\
     then there exist $\Delta$, $\Omega'$, and $A'$
     \\
     such that
     $\substextend{\Delta}{\Omega'}$ and 
     $\substextend{\Omega}{\Omega'}$ and 
     $\synjudg{\Gamma}{e}{A'}{\Delta}$ and 
     $A = [\Omega']A'$. 
\end{quote}

  \begin{itemize}
     \DerivationProofCase{\DeclVar}
          {(x : A) \in [\Omega]\Gamma}
          {\declsynjudg{[\Omega]\Gamma}{x}{A}}
          
          \begin{llproof}
            \inPf{(x : A)} {[\Omega]\Gamma}  {Premise}
            \substextendPf{\Gamma}{\Omega}   {Given}
            \inPf{(x : A')}{\Gamma\text{~where $[\Omega]A' = [\Omega]A$}}   {From definition of context application}
                 \trailingjust{(definition omitted in this example)}
            \LetPf{\Delta}{\Gamma} {}
            \LetPf{\Omega'}{\Omega} {}
\Hand       \substextendPf{\Gamma}{\Omega}   {Given}
\Hand       \substextendPf{\Omega}{\Omega}   {By Lemma 20 (Reflexivity)}
\Hand       \synjudgPf{\Gamma}{x}{A'}{\Gamma}   {By \Var}
            \proofsep
                   \eqPf{[\Omega]A'}{[\Omega]A}   {Above}
\Hand              \continueeqPf{A}   {$\FEV{A} = \emptyset$}
          \end{llproof}
  \end{itemize}

Notes:
\begin{itemize}
\item The line ``(definition omitted in this example)'' is a ``trailing justification'' (\macroname{trailingjust}), for when the justification for a step would be too wide to fit on the page.

\item The last two lines use \macroname{eqPf} and \macroname{continueeqPf}, respectively.

\item The lines marked ``\Pointinghand'' follow my convention for marking conclusions.  Many theorems have the form ``if \dots then
  Q1 and Q2 and \dots''.  These conclusions often aren't all proved at the end
  of the argument, especially when using the induction hypothesis.
  
  In this example, it so happens that they \emph{almost} all come at the end,
  but the final part ($A = [\Omega']A'$ in the statement) needs a transitive
  equality.  Without this convention, I'd either have to repeat all the ``early''
  Q's at the end, or have to scan the entire proof to see if I'd proved
  everything.  This convention also facilitates eliding very minor
  reasoning steps: the line $\substextend{\Gamma}{\Omega}$
  isn't \emph{literally} the desired conclusion $\substextend{\Delta}{\Omega'}$,
  but the presence of ``\Pointinghand'' means that the reader knows that the line
  is supposed to match $\substextend{\Delta}{\Omega'}$, and (hopefully)
  will notice the equalities $\Delta = \Gamma$ and $\Omega' = \Omega$.

  (Deciding which steps are trivial enough to slide under the rug is a
  matter of taste, and I sometimes disagree with my own past decisions.)

  For multi-step equational reasoning with \macroname{continueeqPf},
  my convention is that ``\Pointinghand'' is marking the ``endpoints'' of
  the equality, here $[\Omega]A'$ and $A$, not the last step ($[\Omega]A = A$).

\item The extra space before the ``$[\Omega]A' = [\Omega]A$'' line
  is produced by \macroname{proofsep}.
\end{itemize}

A useful macro for the ``justification'' column, not shown in the above
example, is \macroname{ditto}, which produces \ditto.
For instance, applying the i.h.\ in this
proof kicks out a four-part conclusion (like the four lines marked with
``\Pointinghand''), so I'd give \macroname{ditto} as the last argument
for the second, third, and fourth parts.  To me, \macroname{ditto} has
different semantics than repeating the justification; I wouldn't write
the following, because Lemma 20 only produces one result.

\begin{llproof}
  \substextendPf{\Gamma}{\Gamma}  {By Lemma 20 (Reflexivity)}
  \substextendPf{\Delta}{\Delta}  {\ditto~~~~\textcolor{red}{(not in my idiolect)}}
\end{llproof}

I'm very particular about this kind of thing, but you don't have to be.  I
believe---as usual in discussions of mathematical writing, without any real
evidence---that most of the benefit of \textvtt{llproof} comes from the line-by-line
structure, which makes it much harder to slide \emph{all} the justifications under
the rug, while making it easier to enforce choices about what steps to include or
omit.  For example, I usually don't explicitly note that the i.h.\ is being applied to a
smaller derivation, or whatever.  But if I have
a proof with a weird induction measure, I can easily check whether I'm showing that
it's being applied correctly by looking for ``By i.h.'' lines, and checking that they're preceded by a line saying that something is smaller.


\section{Example figure}

\begin{figure}[htbp]
  \centering
  
  FIGURES FOR THE FIGURE GOD!  CAPTIONS FOR THE THRONE OF CAPTIONS!

  \caption{A caption with \emph{colour}?  Whatever will they think of next?}
  \label{fig:example}
\end{figure}

\end{document}


% Local Variables: 
% mode: latex
% TeX-master: example
% End: 
